%
% Requisitos.tex - padsoft 2014
% (c) 2014  M. Blanc, V. Macias
%

\documentclass[11pt]{article}

% pagina tamaño A4
\usepackage{geometry}
\geometry{letterpaper} 
% español
\usepackage[utf8]{inputenc}
\usepackage[spanish]{babel}
% simbolos de euro
\usepackage[official]{eurosym}
\usepackage{newunicodechar}
\newunicodechar{€}{\euro{}}
% estilos
\usepackage{enumitem}
\usepackage{titlesec}
\newcommand{\sectionbreak}{\clearpage}
\titleformat{\section}{\huge\bfseries}{\thesection}{1em}{}

\title{Documento de análisis de requisitos}
\author{M. Blanc \texttt{manuel.blanc@estudiante.uam.es} \and V. Macias \texttt{victor.macias@estudiante.uam.es}}
\date{Viernes 31 de Enero, 2014}

\begin{document}

\clearpage\maketitle
\thispagestyle{empty}
\tableofcontents

\section{Introducción}
\subsection{Proposito del sistema}
Qué tiene que hacer la aplicación, por qué hay que construirla.

\subsection{Ámbito del sistema}
Resumen de lo que el sistema tiene que hacer y qué cosas no tiene que hacer (hasta dónde llega la funcionalidad).

\subsection{Objetivos y criterios de exito del proyecto}
Objetivos principales del sistema, y criterios que se usarán para decidir si el sistema se ha construido de manera exitosa.

\subsection{Definiciones, acronimos y abreviaturas}
Lista con términos técnicos y abreviaturas que se usarán en el resto del documento.

\section{Descripción del sistema}
\subsection{Requisitos funcionales}
Tipos de usuario y descripción de las funciones que puede realizar cada uno de ellos.
\subsubsection{Tipo de usuario 1}
\subsubsection{Tipo de usuario 2}
\subsubsection{Tipo de usuario 3}

\subsection{Requisitos no funcionales}
Descripción de los requisitos no funcionales, categorizados por tipo (rendimiento, fiabilidad, etc).

\section{Casos de uso}
\subsection{Diagrama de casos de uso}
El diagrama de casos de uso del sistema

\subsection{Descripcion de los casos de uso}
Detalle de los casos de uso del sistema (elige al menos 3). Utiliza la siguiente plantilla:

\subsection{Caso de uso NOMBRE}
\begin{description}[style=nextline]
\item[Actor primario] ...
\item[Interesados y objetivos] ...
\item[Precondiciones] Situación necesaria para que el caso de uso pueda darse
\item[Garantía de exito (postcondiciones)] ...
\item[Escenario principal de éxito] Interacciones del escenario, numeradas
\item[Extensiones (flujos alternativos)] Escenarios excepcionales
\item[Requisitos especiales] Lista de requisitos, probablemente no funcionales
\item[Lista de variaciones de tecnología y datos] Lista de distintas opciones teconologias para la funcionalidad del caso de uso
\item[Frecuencia de ocurrencia] Estimacion (erronea probablemente)
\item[Temas abiertos] Cuestiones que estan abiertas y se plantean para resolucion futura, o para considerarse en futuras versiones
\end{description}

\section{Maquetas}
Aqui van unos preciosos dibujicos

\end{document}

