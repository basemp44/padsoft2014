\section{Introducción}
\subsection{Propósito del sistema}
El cliente ha encargado un paquete de software para administrar un videoclub.
En concreto, se busca automatizar el proceso de gestión de prestamos/devoluciones, proporcionando una interfaz gráfica.
El propósito de este documento es analizar el problema para ayudarnos a entender la situación, y llegar a un acuerdo con el cliente.

\subsection{Ámbito del sistema}
El software está dirigido principalmente a los empleados del videoclub.
El producto final debe permitir efectuar préstamos y devoluciones, y registrar a usuarios nuevos. Lo indispensable en general será:
\begin{itemize}
	\item Dar de alta socios. Para ello se le proporcionarán los datos requeridos, ya descritos, y el software devolverá un UID de socio.
	\item Efectuar préstamos de artículos, prestándolos durante un período de tiempo establecido por el gerente (en principio 3 días). Para ello:
	\begin{itemize}
		\item Se solicitará el UID de socio y se verificará tanto su existencia en la base de datos como que disponga de permisos para el préstamo. 
		\item Se buscará el artículo pedido. Para ello se ingresará el tipo de artículo y se buscará.
		\item Tras ello, se generará una transacción de compra en la que se seleccionará el tipo de pago, que podrá ser en efectivo o con tarjeta, y el socio deberá abonar el importe.
	\end{itemize}
	\item Vender contratos de tarifa plana de 30 días naturales. Un socio podrá comprar contratos de 1, 2 y 3 tipos de artículo distintos, que serán excluyentes para dicho socio. Además se podrá pagar un plus si el usuario quiere tener un día más de tarifa plana.
	\item Cambiar la disponibilidad de artículos deteriorados.
	\item Procesar devoluciones, verificando si el artículo a devolver tiene sanción. En caso de tener sanción, el software deberá generar una transacción con el importe a abonar. Hay dos niveles de sanción: sanción a corto plazo (antes de 10 días) y a largo plazo (a partir del día 10), con precios diferentes gestionados por el gestor.
\end{itemize}

Hemos considerado una serie de caracteristicas --no indispensables-- que pueden aportar valor adicional al producto final:
\begin{itemize}
	\item Contabilidad de la caja, quedando registro de todos los trámites en la aplicación. Esta mejora supone poder visualizar estos datos.
	\item Control de desperfectos. Es conveniente tener las sanciones unificadas en el sistema de devoluciones.
	\item Dar de baja socios. Los socios deberían poder solicitar que se borren sus datos de carácter personal de la base de datos.
\end{itemize}

Asimismo, tambien se ha acordado hacer ciertas omisiones en la funcionalidad del software para simplificarlo y reducir gastos.
\begin{itemize}
	\item Venta de películas viejas.
	\item Reservas. Es preferible que se gestionen por otra vía.
	\item Configuración del IVA. Los precios pueden ser ajustados manualmente y complica la interfaz.
\end{itemize}

%El fin del programa es que sea usado por los empleados del videoclub para asistirles con su trabajo.
%El sistema debe estar diseñado con esto en mente. Las labores propias de los empleados, en concreto efectuar préstamos y devoluciones. funcionamiento independiente.
%El sistema debe permitir controlar el stock y los prÉstamos. Además, debe permitir al gestor del videoclub obtener estadísticas sobre los clientes.
%El programa final debe ser independiente, y no deberá ser recopilado para añadir peliculas nuevas a la base de datos.

\subsection{Objetivos y criterios de éxito del proyecto}
El objetivo principal es facilitar la labor de los empleados.
Se considerara que el software ha alcanzado un nivel de funcionalidad básico cuando se puedan realizar las siguientes tareas dentro de un marco de tiempo razonable.
\begin{itemize}
	\item Registrar a un usuario nuevo en la base de datos.
	\item Efectuar un alquiler, denegando la operación si los datos necesarios no son válidos.
	\item Reiniciar la aplicación y comprobar que los datos son persistentes entre ejecuciones.
	\item Efectuar una devolución que lleve un retraso, y calcular correctamente la tasa de atraso.
\end{itemize}

\subsection{Definiciones, acrónimos y abreviaturas}
La aplicación debe representar una serie de objetos internamente.
A continuación estan definidas los campos que forman parte de cada una de las entidades lógicas.
Estos objetos los definimos como una tupla ordenada de los atributos que lo componen.

\subsubsection{Socios}
Se debe guardar toda la información necesaria para identificar a los socios, efectuar pagos y ponerse en contacto con ellos. Esta información es confidencial y debería estar guardada de manera segura.
\begin{description2}
	\item[UID]              Número de socio, aparecere en los carnéts emitidos (10 dígitos).
	\item[Nombre]           Nombre y apellidos del socio.
	\item[DNI/NIE]          Número de DNI completo, o alternativamente el NIE.
	\item[Dirección]        Dirección completa.
	\item[Teléfono]         Número de teléfono, móvil o fijo.
	\item[Email]            Dirección de correo electrónico. \textit{Opcional.}
\end{description2}

\subsubsection{Películas}
\begin{description2}
	\item[Título]           Título de la película.
	\item[Fecha]            Fecha de publicación.
	\item[Categorías]       Lista de categorías a las que pertenece.
	\item[Director]         Nombre del director.
	\item[Formato]          Formato de la película, puede ser DVD o Bluray.
	\item[Num. ejemplares]  Número de ejemplares \emph{total}.
\end{description2}

\subsubsection{Series}
\begin{description2}
	\item[Título]           Título completo de la serie.
	\item[Temporada]        Número de temporada.
	\item[Categorias]       Lista de categorias a las que pertenece.
	\item[Volumen]          Número de volumen. 
	\item[Formato]          Formato de la serie, puede ser DVD o Bluray.
	\item[Num. ejemplares]  Número de ejemplares \emph{total}.
\end{description2}

\subsubsection{Música}
\begin{description2}
	\item[Título]           Título completo del disco.
	\item[Géneros]          Géneros musicales a los que pertenece el disco.
	\item[Interprete]       Lista de interpretes, separada por comas.
	\item[Año]              Año de publicacion.
	\item[Formato]          Formato del disco, puede ser CD o vinilo.
	\item[Num. ejemplares]  Número de ejemplares \emph{total}.
\end{description2}

