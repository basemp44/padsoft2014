\section{Descripción del sistema}
\subsection{Requisitos funcionales}
Se debe hacer una distinción entre los empleados y el gerente del establecimiento.
Integracion con el SO.

\subsubsection{Tipo de usuario I: Empleado}
La interfaz de los empleados debe tener, obligatoriamente, al menos los siguientes elementos:
\begin{itemize}
	\item Un menú desde el que se pueda escoger el resto de opciones.
	\item Pantalla para dar de alta y baja socios.
	\item Pantalla de busqueda.
	\item Pantalla de pagos.
	\item Pantalla para hacer préstamos, debe tener búsqueda integrada.
	\item Pantalla de contratar tarifas VIP.
	\item Cambiar la disponibilidad de artículos deteriorados.
	\item Pantalla para devoluciones.
\end{itemize}

\subsubsection{Tipo de usuario II: Gerente}
El gerente de la tienda debe poder administrar la tienda desde la aplicación. Para acceder a esta funcionalidad, se debe proporcionar una contraseña. Su interfaz debe contener:
\begin{itemize}
	\item Pantalla para modificar los precios de alquiler de todas las categorías.
	\item Pantalla para crear y eliminar artículos nuevos en la base de datos.
	\item Visualización de la lista de morosos.
	\item Visualización de la lista de los articulos más solicidatos (top-ten).
\end{itemize}

\subsection{Requisitos no funcionales}
\subsubsection{Portabilidad}
El cliente puso especial énfasis es que la aplicación debe estar hecha en Java\texttrademark, y funcionar sobre Microsoft Windows\texttrademark.

\subsubsection{Accesibilidad}
Los empleados de la tienda no tienen por qué tener experiencia con ordenadores, y no deberían necesitar preparación para usar el software.

\subsubsection{Estabilidad}
El establecimiento maneja un volúmen considerable de alquileres mensualmente. El software debe poder almacenar un volúmen de datos considerable,aproximadamente 10.000 artículos entre las 3 categorias.

Además, el software debe garantizar la integridad de los datos en casos de uso normales.
En caso de error, el sistema debe manejarlo elegantemente.


\subsubsection{Seguridad}
Debe haber una jerarquía de permisos tal que los empleados no puedan acceder a acciones privilegiadas del gerente.
